\documentclass[14pt]{extarticle}
%------байда для русского текста в формулах
\usepackage{mathtext}         % если нужны русские буквы в формулах (не обязательно)
\usepackage[T2A]{fontenc}     % внутренняя T2A кодировка TeX
\usepackage[utf8]{inputenc} % кодировка - можно использовать [cp866] [cp1251]
%------до седова байда
\usepackage[russian,english]{babel}
\usepackage{amsmath}
\usepackage{geometry} % Меняем поля страницы
\geometry{left=1cm}% левое поле
\geometry{right=2cm}% правое поле
\geometry{top=2cm}% верхнее поле
\geometry{bottom=2cm}% нижнее поле
\usepackage{graphicx}%рисунки
%\usepackage{pgf}
\usepackage{pgfplotstable}
%\usepackage{pscyr}\usepackage{colordvi}
\usepackage{listings}

\usepackage{longtable}

%для того чтобы не писать свое фио и группу везде
\newcommand{\mygroup}{студент группы ПВТ-Д-4}
\newcommand{\myname}{Барышников Н. М.}
%имя преподавателя
\newcommand{\prname}{ст. преподаватель Денисов А. А. }
%дисциплина
\newcommand{\discip}{Параллельные системы и параллельное программирование}
%номер лабы
\newcommand{\numlab}{1}
%тема
\newcommand{\labtheme}{Поиск максимального элемента массива с использованием нескольких thread'ов}
%\renewcommand{\labelenumii}{\arabic{enumi}.\arabic{enumii}.}%страшное переопределение счетчика для %того чтобы вложенные списки были вида 1 1.1 1.2 1.3/2 2.1 2.2 а не 1 a b c 2 a b c
%\renewcommand{\theenumi}{\Asbuk{enumi})}%нумеруем списки буквами
\renewcommand{\labelenumi}{\theenumi.}
%далее идет фигня про колонтитулы
%\usepackage{fancyhdr}%ибо не врубился как переопределять стандартные
%\fancyhf{}%сбиваю настройки пакета
%\pagestyle{fancy}%выбираю стить
%\fancyhead[RO]{\myname}%верхний колонтитул надпись справа
%\fancyhead[LO]{}%верхний колонтитул надпись слева
%\fancyfoot[CO]{\bfseries\thepage}%нижний колонтитул надпись по центру(номера страниц)
%\renewcommand{\headrulewidth}{0.5pt}
%\renewcommand{\footrulewidth}{0pt}
%\addtolength{\headheight}{0.5pt} % оставляем место для линейки
%\fancypagestyle{plain}{%
%  fancyhead{} % на обычных страницах убираем колонтитулы
%   renewcommand{\headrulewidth}{0pt} % и линейку
%}
  % убираем текущие установки для колонтитулов
  
\begin{document}
\section{Организационная структура предприятия}
Открытое акционерное общество <<Медицина>> представляет собой многопрофильный медицинский центр, включающий:
\begin{itemize}
\item поликлинику;
\item стационар;
\item травмпункт;
\item круглосуточную скорую медицинскую помощь;
\item онкологический центр Sofia.
\end{itemize}
Персонал ОАО <<Медицина>> делится на 2 большие части: медицинский и немедицинский персонал.
Медицинский персонал составляют сотрудники различных отделений клиники и стационара.\\
Отделения поликлиники:
\begin{itemize}
\item клиническое отделение;
\item диагностическое отделение;
\item отделение семейной медицины;
\item стационар;
\item стоматологическое отделение;
\item отделение восстановительной медицины;
\item онкологический центр Sofia;
\end{itemize}
Немедицинский персонал составляет различные службы и подразделения обслуживающие клинику.
Немедецинские службы:
\begin{itemize}
\item служба инженерного обслуживания;
\item служба информационных технологий;
\item отдел рекламы и маркетинга;
\item отдел бухгалтерского учета;
\item отдел кадров;
\item отдел по работе с клиентами;
\end{itemize}
Практика проходила в службе информационных технологий, в должности практиканта службы информационных технологии, подробнее структура этой службы будет рассмотрена в разделе~\ref{sec:ITSRV}.
\subsection{Служба информационных технологий} \label{sec:ITSRV}
Служба информационных технологий занимается всей информационной структурой медицинского центра. Она обеспечивает работу сети организации и ее администрирования, занимается обеспечением клиентов центра доступом к информационном услугам, таким как wi-fi на всей территории, IP-TV для стационара, поддержка информационных ресурсов в сети Интернет, поддержкой программных систем, используемых клиникой (например поддержкой системы бухгалтерского учета и системы электронной истории болезни), так же служба занимается обеспечением безопасности персональных данных клиентов и работников клиники, занимаясь поддержкой системы СКУД, резервирования информации, правильной настройки прав доступа для всех пользователей сети и программных систем. Кроме того, служба включает в себя call-центр, который осуществляет поддержку пользователей и взаимодействие со страховыми компаниями.

Работа службы построена по модели ITSM. Это модель взаимодействия с бизнесом подразумевает, что само предприятие взаимодействует с IT-отделом как с поставщиком услуг для бизнеса. Это позволяет сосредоточиться на качестве предоставляемых услуг, а также гибко изменять используемые технологии. 

Все взаимодействие службы и организации построено через систему Service Desk, куда подаются заявки отделов и сотрудников организации. Все заявки и работы IT-отдела формализованы, будь то починка чего-либо или написание ПО. Это позволяет довольно быстро реагировать на заявки, а так же довольно точно определять время и ресурсы, которые будет затрачены на обслуживание клиента.
\section{Методы и средства решения поставленных задач на практику}
\subsection{Язык C\#}
Важной частью практики было изучение языка С\# .  Это объектно ориентированный язык программирования,созданный компанией Microsoft.  Программирование на данном языке базируется на применении программных компонентов в форме автономных и самодокументируемых функциональных модулей. Основной особенностью таких компонентов является реализация модели программирования с использованием свойств, методов, событий и атрибутов, представляющих декларативное описание компонентов, а также включение в них собственной документации. В C\# представлены языковые конструкции, непосредственно поддерживающие эти понятия, что делает его близким к естественному языком для создания и применения программных компонентов.

В C\# применяются унифицированная система типов. Все типы C\# наследуются от единственного типа object. Для всех типов существует набор общих операций, например метод ToString(). Пользователь также может создавать свои типы.

Программа на C\# представляет собой класс. Класс имеет один обязательный метод Main, который выполняется при запуске программы. 

Разработчик также может добавлять собственные поля, методы, свойства, классы

Свойства представляют собой естественное расширение полей. Свойства -- член класса, он представляет собой поле с методами доступа к нему. Существует два метода доступа get и set. Get используется для доступа к значению связанного со свойством, а set используется для установки значения свойства.

Свойства очень удобны для создания членов классов с контролем пустого значения. 

\noindent Например:
\lstset{
extendedchars=\true,
inputencoding=utf8,
language=[Sharp]c
}
\begin{lstlisting}
public class Example{
	private	ExampleObject x;

	public ExampleObject ExampleobjectProperty{
		get{
		 if(ExamleObject==NULL) x=new ExampleObject();
		 return x;
		}
		set{
		 ExamleObject=value;
		}	
	}
}
\end{lstlisting}

\bigskip

Делегат -- это особый тип, который представляет собой ссылку на метод с конкретным списком параметров и типом возвращаемого значения. Понятие делегата близко к понятию указателя на функцию, используемому в некоторых других языках, напримерё~С. 

\noindent Пример делегата:
\begin{lstlisting}
delegate double Func(double x);
Func fun=Math.Sin;
double result=fun(2.5);//result=0.598412144103957...
fun=Math.Abs;
result=fun(2.5);//result=2.5;
\end{lstlisting}

\bigskip

События это члены класса, который используется классом для уведомлений программы о возникновении какой-либо ситуации, обработку которой разработчик класса предоставил пользователю класса.

События позволяют разработчику назначать функции, которые будут вызываться в ситуации, в которой было порождено событие. Эти функции называются обработчиками событий.


\subsection{Система контроля версий Git}
После потери нескольких важных файлов, было решено воспользоваться системой контроля версий. Была выбрана система Git 
\end{document}