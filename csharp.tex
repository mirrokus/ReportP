\section{Методы и средства решения поставленных задач на практику}
\subsection{Язык C\#}
Важной частью практики было изучение языка С\# .  Это объектно ориентированный язык программирования,созданный компанией Microsoft.  Программирование на данном языке базируется на применении программных компонентов в форме автономных и самодокументируемых функциональных модулей. Основной особенностью таких компонентов является реализация модели программирования с использованием свойств, методов, событий и атрибутов, представляющих декларативное описание компонентов, а также включение в них собственной документации. В C\# представлены языковые конструкции, непосредственно поддерживающие эти понятия, что делает его близким к естественному языком для создания и применения программных компонентов.

В C\# применяются унифицированная система типов. Все типы C\# наследуются от единственного типа object. Для всех типов существует набор общих операций, например метод ToString(). Пользователь также может создавать свои типы.

Программа на C\# представляет собой класс. Класс имеет один обязательный метод Main, который выполняется при запуске программы. 

Разработчик также может добавлять собственные поля, методы, свойства, классы

Свойства представляют собой естественное расширение полей. Свойства -- член класса, он представляет собой поле с методами доступа к нему. Существует два метода доступа get и set. Get используется для доступа к значению связанного со свойством, а set используется для установки значения свойства.

Свойства очень удобны для создания членов классов с контролем пустого значения. 

\noindent Например:
\lstset{
extendedchars=\true,
inputencoding=utf8,
language=[Sharp]c
}
\begin{lstlisting}
public class Example{
	private	ExampleObject x;

	public ExampleObject ExampleobjectProperty{
		get{
		 if(ExamleObject==NULL) x=new ExampleObject();
		 return x;
		}
		set{
		 ExamleObject=value;
		}	
	}
}
\end{lstlisting}

\bigskip

Делегат -- это особый тип, который представляет собой ссылку на метод с конкретным списком параметров и типом возвращаемого значения. Понятие делегата близко к понятию указателя на функцию, используемому в некоторых других языках, напримерё~С. 

\noindent Пример делегата:
\begin{lstlisting}
delegate double Func(double x);
Func fun=Math.Sin;
double result=fun(2.5);//result=0.598412144103957...
fun=Math.Abs;
result=fun(2.5);//result=2.5;
\end{lstlisting}

\bigskip

События это члены класса, который используется классом для уведомлений программы о возникновении какой-либо ситуации, обработку которой разработчик класса предоставил пользователю класса.

События позволяют разработчику назначать функции, которые будут вызываться в ситуации, в которой было порождено событие. Эти функции называются обработчиками событий.


\subsection{Система контроля версий Git}
После потери нескольких важных файлов, было решено воспользоваться системой контроля версий. Была выбрана система Git 