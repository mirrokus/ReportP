\section{Организационная структура предприятия}
Открытое акционерное общество <<Медицина>> представляет собой многопрофильный медицинский центр, включающий:
\begin{itemize}
\item поликлинику;
\item стационар;
\item травмпункт;
\item круглосуточную скорую медицинскую помощь;
\item онкологический центр Sofia.
\end{itemize}
Персонал ОАО <<Медицина>> делится на 2 большие части: медицинский и немедицинский персонал.
Медицинский персонал составляют сотрудники различных отделений клиники и стационара.\\
Отделения поликлиники:
\begin{itemize}
\item клиническое отделение;
\item диагностическое отделение;
\item отделение семейной медицины;
\item стационар;
\item стоматологическое отделение;
\item отделение восстановительной медицины;
\item онкологический центр Sofia;
\end{itemize}
Немедицинский персонал составляет различные службы и подразделения обслуживающие клинику.
Немедецинские службы:
\begin{itemize}
\item служба инженерного обслуживания;
\item служба информационных технологий;
\item отдел рекламы и маркетинга;
\item отдел бухгалтерского учета;
\item отдел кадров;
\item отдел по работе с клиентами;
\end{itemize}
Практика проходила в службе информационных технологий, в должности практиканта службы информационных технологии, подробнее структура этой службы будет рассмотрена в разделе~\ref{sec:ITSRV}.
\subsection{Служба информационных технологий} \label{sec:ITSRV}
Служба информационных технологий занимается всей информационной структурой медицинского центра. Она обеспечивает работу сети организации и ее администрирования, занимается обеспечением клиентов центра доступом к информационном услугам, таким как wi-fi на всей территории, IP-TV для стационара, поддержка информационных ресурсов в сети Интернет, поддержкой программных систем, используемых клиникой (например поддержкой системы бухгалтерского учета и системы электронной истории болезни), так же служба занимается обеспечением безопасности персональных данных клиентов и работников клиники, занимаясь поддержкой системы СКУД, резервирования информации, правильной настройки прав доступа для всех пользователей сети и программных систем. Кроме того, служба включает в себя call-центр, который осуществляет поддержку пользователей и взаимодействие со страховыми компаниями.

Работа службы построена по модели ITSM. Это модель взаимодействия с бизнесом подразумевает, что само предприятие взаимодействует с IT-отделом как с поставщиком услуг для бизнеса. Это позволяет сосредоточиться на качестве предоставляемых услуг, а также гибко изменять используемые технологии. 

Все взаимодействие службы и организации построено через систему Service Desk, куда подаются заявки отделов и сотрудников организации. Все заявки и работы IT-отдела формализованы, будь то починка чего-либо или написание ПО. Это позволяет довольно быстро реагировать на заявки, а так же довольно точно определять время и ресурсы, которые будет затрачены на обслуживание клиента.